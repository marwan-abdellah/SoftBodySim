\chapter{Wstęp} %rozpoczêcie pracy nowym rozdzia³em

Modelowanie ciał deformowalnych od prawie 30 lat zajmuje szczególne miejsce 
w obszarze symulacji komputerowych. Naturalnie wyglądające animacje tkanin, 
płynów czy włosów, dodają symulacji walorów wizualnych, 
która np. w przypadku gier czy filmów, jest jednym z czynników decydujących o sukcesie 
komercyjnym produktu. Ekonomiczne przesłanki skutkują ciągłym powstawaniem 
nowych metod symulacji ciał miękkich oraz ich implementacją w praktycznie każdym
współczesnym silniku fizycznym, takim jak PhysiX czy Bullet.

