\chapter{Wstęp}

% czemu ten temat jest interesujący

% dlaczego ta praca

% co z niej można wynieść

% opis rozdziałów

Symulacje ciał miękkich od zawsze stanowiły przedmiot szczególnego
zainteresowania twórców gier. Realistycznie wyglądające symulacje materiałów,
włosów czy możliwość deformowania obiektów przez gracza podnoszą atrakcyjność
tworzonych gier, co nierzadko decyduje o komercyjnym sukcesie produktu.
Nie dziwi zatem nakład środków ponoszony na rozwój technik symulacji ciał
miękkich, czego efektem jest powstawianie coraz to bardziej zadziwiających
efektów wizualnych w najnowszych grach i symulacjach komputerowych.

Praca ta ma za zadanie wprowadzić czytelnika w bogaty świat metod symulacji ciał
miękkich. W rozdziale drugim zostały podjęte próby usystematyzowania
dotychczasowych osiągnięć w tej dziedzinie oraz pokazania w jakich obszarach
stosowane są dane typy symulacji. Rozdział ten nie miał jednak na celu
dokładnego omówienia sposobów działania wszystkich metod, lecz tylko pobieżny
przegląd aktualnego stanu wiedzy i skupienie się na najpopularniejszym
stosowanym modelu - Systemie Sprężyn i Punktów Mas.

W rozdziale drugim przedstawione zostaną też prawa fizyki mające zastosowanie
w opisie ciał deformowalnych. Ze względu jednak na obszerność tego zagadnienia,
niemożliwe będzie ich wyczerpujące opisanie. Zamiast tego nacisk zostanie
położony na omówienie praw fizycznych mających zastosowanie w konkretnych
modelach opisanych w dalszej części tego rozdziału.

Spośród wielu dostępnych technik symulacji ciała miękkiego zostały wybrane i
opisane w rozdziale trzecim dwie relatywnie nowe i proste w implementacji
techniki. Pomimo prostoty przedstawione w pracy metody są podstawą współczesnych
symulacji ciał miękkich i stanowią bazę do tworzenia bardziej skomplikowanych i
wyspecjalizowanych modeli używanych w zaawansowanych silnikach graficznych
takich jak NVIDIA PhysX, Havoc Cloth, Maya nCloth czy Bullet \cite{Liu:2013:FSM}.

Rozdział czwarty stanowi wstęp do technologii CUDA, która dzisiaj jest 
powszechnie wykorzystywana do wspierania obliczeń fizycznych w symulacjach
komputerowych. Zostały w nim omówione zmiany jakie zaszły na przestrzeni
ostatnich lat w dziedzinie programowani kart graficznych, które umożliwiły ich
zastosowanie nie tylko do procesu generowania obrazu. Rozdział ten stanowi też
podsumowanie najważniejszych informacji jakie każdy programista CUDA powinien
wiedzieć.

Integralną częścią stworzonej pracy jest aplikacja demonstracyjna implementujące
techniki omówione w rozdziale trzecim. Jej opis wraz z problemami jakie
napotkałem w czasie implementacji zawarty jest w rozdziale piątym. Ostatni
rozdział zawiera opis technik optymalizacji stworzonej aplikacji ze szczególnym
uwzględnieniem kodu CUDA przeznaczonego do wykonania na procesorze graficznym.
