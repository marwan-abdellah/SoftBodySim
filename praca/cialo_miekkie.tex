\chapter{Ciało miękkie w symulacji komputerowej} 

Modelowanie zjawisk fizycznych od wielu lat stanowi przedmiot szczególnego
zainteresowania branży gier komputerowych. Naturalnie wyglądające animacje tkanin, płynów, oświetlenia, włosów, skóry, wszelkiego rodzaju ciał mogących doznać odkształceń w wyniku interakcji z użytkownikiem, niewątpliwie dodają symulacji walorów wizualnych. Taka wizualna atrakcyjność, np. w przypadku produkcji gier, często okazuje się jednym z głównych czynników decydujących o sukcesie komercyjnym produktu. Nie dziwi zatem pokaźna liczba ludzi i nakładów zaangażowanych w badania nad sposobami symulacji zjawisk fizycznych, czego efektem są liczne publikacje, również na temat symulacji ciał deformowalnych.

Wiele ze stworzonych dotychczas metod symulacji ciał miękkich za priorytet stawia  stabilność, responsywność oraz szybkość symulacji, często kosztem dokładności obliczeń oraz poprawności modelu fizycznego. Implikuje to fakt, że powstałe modele stają się mocno wyspecjalizowane, zdolne do symulacji tylko jednego wąskiego spektrum zjawisk fizycznych, jak np. symulacja tkanin, czy włosów. Ponadto, oprócz zawężenia przedmiotu symulacji, zawęża się też jej zakres zastosowań. Z racji niedokładności obliczeń nie można owych modeli wykorzystać do obliczeń inżynierskich, więc znajdują one główne zastosowanie w grach komputerowych, czy ogólniej - animacji komputerowej. \cite{pbdyn}

Jedną z najczęściej wykorzystywanych technik służących do symulacji ciała miękkiego jest operowanie na siłach działających na ciało. Schemat takiej symulacji można uogólnić, zakładając, że ciało miękkie przedstawiamy jako zbiór punktów na które mogą działać siły. W każdym kroku akumulowane są siły zewnętrzne i wewnętrzne działające na każdy punkt modelu. Jako siły wewnętrzne najczęściej wymieniane są siły sprężystości czy ciśnienia, jako siły zewnętrzne siła grawitacji czy siły powstałe w następstwie kolizji z innymi obiektami. Następnie w każdym kroku symulacji z sił wyliczane jest przyspieszenie punktów zgodnie z drugim prawem dynamiki Newtona. W kolejnych krokach, wykorzystując dowolną metodę, całkujemy otrzymany układ sił obliczając prędkości, a następnie nowe pozycje punktów modelu.\cite{pbdyn}

Przykładem modelu z w/w rodziny jest omawiany w następnych podrozdziale - system sprężyn i punktów masy, (ang. mass-spring system). W kolejnych podrozdziałach zostaną omówione dodatkowe, wybrane z literatury przedmiotu obostrzenia modelu, mające na celu zniwelować wiele niepożądanych efektów symulacji.

\section{System Punktów Masy i Sprężyn'}
W tym rozdziale zostanie scharakteryzowana jedna z najpowszechniejszych metod symulacji ciała miękkiego. Jak wskazuje nazwa modelu system składa się z systemu  dwóch podstawowych elementów:
\begin{itemize}
\item  Punkt Masy - punkt w przestrzeni posiadające masę, na który mogą oddziaływać siły.
\item Sprężyna - rozciągnięta pomiędzy dwoma punktami masy, posiada swoją normalną długość, nie posiada masy.

\end{itemize} 

% sześcian 2x2x2 składający się z punktów masy i sprężyn między nimi
\begin{figure}[ht]
\centering
\begin{tikzpicture}

    \draw[-,snake=coil] (0,0 ,0) -- (0,3 ,0);
    \draw[-,snake=coil] (0,0 ,3) -- (0,3 ,3);
	\draw[-,snake=coil] (3,0 ,0) -- (3,3 ,0);
	\draw[-,snake=coil] (3,0 ,3) -- (3,3 ,3);
	\draw[-,snake=coil] (0,0 ,0) -- (3,3 ,3);
	\draw[-,snake=coil] (3,0 ,3) -- (0,3 ,0);
\foreach \y in{0,3}
{
    \draw[-,snake=coil] (0,\y ,0) -- (3,\y ,0);
    \draw[-,snake=coil] (0,\y ,3) -- (3,\y ,3);
	\draw[-,snake=coil] (0,\y ,0) -- (0,\y ,3);
	\draw[-,snake=coil] (3,\y ,0) -- (3,\y ,3);

	\filldraw[fill=red, draw=black] (0, \y, 0) circle (5pt);
	\filldraw[fill=red, draw=black] (3, \y, 0) circle (5pt);
	\filldraw[fill=red, draw=black] (0, \y, 3) circle (5pt);
    \filldraw[fill=red, draw=black] (3, \y, 3) circle (5pt);
}

\end{tikzpicture}

\caption{Zbiór punktów masy z przykładowym układem połączeń}
\end{figure}
Model ten posiada podstawy fizyczne, ponieważ siły generowane przez rozciągniętą sprężynę są zgodne z prawem Hook'a. W swoim podstawowym wariancie siły działające na jeden punkt masy układu są zdefiniowane jako:

%
% Równanie ogólne siły działającej na punkt masy
%
\begin{equation}
F_{i} = \sum_{j} g_{ij} + f^{d}_i + f^{ex}_{i}
\end{equation}

W powyższym równaniu na punkt masy w danej chwili $t$ działają siły:
\begin{itemize}
\item  Sprężystości $g_{ji}$ generowane przez sprężyny zawieszone między sąsiadującymi punktami.

\begin{equation}
g_{ij} = k_s (| x_{ij}| - l_{ij})\frac{x_{ij}}{|x_{ij}|}
\end{equation}
,gdzie $x_{ij} = x_i - x_j$, jest wektorem różnicy położeń między sąsiadującymi punktami masy. Siła sprężystości w modelu jest zgodna z prawem Hook'a, czyli jest proporcjonalna do odchylenia sprężyny od jej spoczynkowej długości $l_{ij}$. Współczynnik $k_s$ jest współczynnikiem sprężystości i z założenia jest zależny od materiału z którego składa się symulowane ciało.

\item Tłumienia $f^{d}_i$ wynikająca z faktu, iż symulowanie ciało nie jest doskonale elastyczne i nie zachowuje energii układu. (Tzn. energia mechaniczna jest transformowana w energię wewnętrzną ciała, jednak z punktu widzenia symulacji energia nie jest zachowana.)

\begin{equation}
f^{d}_i = k_d(v_j - v_i)
\end{equation}
,gdzie $v_i$ i $v_j$ są wektorami prędkości dwóch punktów masy połączony sprężynami, a $ k_d$ jest współczynnikiem tłumienia charakterystycznym dla symulowanego materiału.

\item Zewnętrzne $f^{ex}_{i}$ działające na punkt materialny, takie jak np. grawitacja.
\end{itemize}. 

Zdefiniowany model jest w istocie równaniem różniczkowym drugiego rzędu i może
być rozwiązany jednym z wielu algorytmów numerycznych. Jedną z najczęściej
wykorzystywanych metod jest algorytm Verleta, który cechuje się prostotą, dając
jednocześnie wystarczająco dokładne rozwiązania. Badania przeprowadzone w \cite{var} pokazały, że algorytm Verleta okazał się najwydajniejszy w porównaniu z innymi metodami numerycznymi, dlatego też będzie stosowany w niniejszej pracy.

Wzór na pozycję punku masy w czasie $t + dt$ jest w modelu wyrażona wzorem:

% Wzór na dynamikę punktu w modelu (Verlet)
\begin{equation}
x_i(t + dt) = \frac{F_i(t)}{m} dt^2 + 2x_i(t) - x_i(t - dt)
\end{equation}

Ze względu na fakt, iż niektóre siły są zależne od poprzednich prędkości punktu masy, do symulacji wykorzystywane będzie też wariant prędkościowy algorytmu Verleta. Jest on zapisany wzorem:

% Wzór na dynamikę punktu w modelu (prędkościowy Verlet)
\begin{eqnarray}
x_i(t + dt) = \frac{F_i(t)}{2m} dt^2 + x_i(t) + v_i(t)dt \\
v_i(t + dt) = \frac{F_i(t + dt) + F_i(t)}{2m}dt + v_i(t)
\end{eqnarray}

%
% MODYFIKACJE MODELU
%
\subsection{Modyfikacje modelu}

\subsubsection{Siła tłumienia}
Siła tłumienia zdefiniowana jako różnica prędkości między dwoma punktami masy w podstawowym modelu jest rzadko stosowana, ze względu na wiele niepożądanych własności. Tłumi ona np. obrót ciała wokół nieruchomego punktu masy, jak przedstawiono na rys. \ref{tlumienie}.

\begin{figure}[ht]
\centering
\begin{tikzpicture}

\draw[->,dotted,] (2.82,0) arc (0:120:2.82) ;
\draw[-,snake=coil] (0,0) -- (2,2);
\draw[->, thick] (2,2) -- (1,3);
\filldraw[fill=red, draw=black] (0,0) circle (5pt);
\filldraw[fill=red, draw=black] (2,2) circle (5pt);
\draw (1.8, 2.6) node {$v_i$};
\end{tikzpicture}


\caption{Rotacja wokół nieruchomego punktu}
\label{tlumienie}
\end{figure}

Według \cite{pbdo} przyjęcie prostej różnicy prędkości punktów w przypadku symulacji materiałów tłumi ich pożądane własności, takie jak podatność na gięcie i marszczenie. Dlatego też warunek na siłę tłumienia zdefiniujemy jako:
\begin{equation}
f^{d}_i = k_d (\frac{v_{ij}^\intercal x_{ij}}{x_{ij}^\intercal x_{ij}}) x_{ij}
\end{equation}

,gdzie $v_{ij} = v_i - v_j$. Powyższe równanie wyznacza siłę tłumienia modelu równą iloczynowi współczynnika tłumienia i projekcji różnicy prędkości dwóch punktów masy na wektor ich różnicy położeń. Definicja nakłada zatem ograniczenie, iż siła tłumienia może działać tylko w tym samymi kierunku co wektor różnicy położeń.

\subsubsection{Zachowanie objętości}
Kolejnym, istotnym aspektem symulacji ciała miękkiego jest zachowanie jego objętości. System punktów mas i sprężyn nie symuluje obiektów posiadających objętość, także często może się zdarzyć, że układ znajdzie się w stanie stabilnym, jednak różnym od wyjściowego. W praktyce często oznacza to, że w wyniku działania dużych sił elementy modelu zostaną obrócone lub zapadną się w swoją własną strukturę. Przykład tego jest przedstawiony na rysunku \ref{stany} sześcian 

\begin{figure}[ht]
\centering
\includegraphics[width=7cm, height=7cm]{images/stabilny.png}
\includegraphics[width=7cm, height=7cm]{images/niestabilny.png}
\caption{Dwa stany stabilne dla sześciennego modelu.}
\label{stany}
\end{figure}

Rozwiązaniem problemu przechodzenia układu między stanami stabilnymi okazało się wprowadzenie sztucznej siły, pozwalającej zachować objętość. Takie podejście po raz pierwszy zaproponowano w \cite{rmofa}. Autorzy publikacji pogrupowali znajdujące się w układzie punkty masy w obiekty dla których można było zdefiniować objętość. Następnie w zależności od różnicy pomiędzy objętością spoczynkową a aktualną generowana była siła oddziałująca punkty masy. Kierunek tej siły jest zgodny z działaniem pewnej z góry zdefiniowanej normalnej. W \cite{isodb} autorzy przedstawiają bardziej ogólny przypadek przyjmując, że obiektem posiadającym objętość jest czworościan.Wierzchołki figury są punktami masy, a krawędzie sprężynami. Siła zachowawcza działająca na dany punktu masy $i$ czworościanu, określa się wzorem:

\begin{equation}
F_i^d = d_v ( v - v_0) n_i
\end{equation}
,gdzie $v$ jest aktualną objętością symulowanego czworościanu, $v_0$ jest jego spoczynkową objętością a $d_v$ jest arbitralnie zdefiniowaną stałą. $n_i$ jest to normalna przeciwległej ściany czworościanu. Podana metoda pozwala uniknąć odwrócenia wierzchołków symulowanego obiektu, ponieważ w takim przypadku obliczona objętość będzie ujemna i powstała, duża siła $F_i^d$ wymusi powrót układu do stanu wejściowego \cite{isodb}.

\subsubsection{Zależność od topologii}
W analizowanym modelu topologia połączeń między punktami mas jest z góry zdefiniowana. Można powiedzieć, że jest to kolejny parametrem symulacji, który w istotny sposób decyduje o jej jakości. Takie założenie samo w sobie nie jest błędne, gdyż modelując wewnętrzną strukturę ciała możemy określić jego fizyczną charakterystykę. Na przykład, symulując elastyczny sześcian i dodając dodatkowe połączenia między punktami masy w jednej płaszczyźnie otrzymamy obiekt różnie podatny na odkształcanie w zależności od kierunku działania siły. Taką mechaniczną właściwość ciała nazywany anizotropią. Anizotropia stanowi bardzo ciekawy przykład własności mechanicznej materiału, której implementacji w modelu punktów mas i sprężyn jest często problematyczna. 

Idealny model powinien umożliwiać symulowanie materiałów izotropowych (o własnościach mechanicznych niezależnych od kierunki działań siły) jak i anizotropowych. Poprzez możliwość manipulacji rozmieszczeniem punktów materialnych, sposoby ich połączeń sprężynami czy manipulowanie stałymi sprężystości, model dostarcza narzędzi do implementacji tych własności. Nie mniej jednak niektóre własności mogą pojawiać się wbrew wcześniejszym założeniom. Przykład niepożądanej anizotropii, otrzymanej poprzez różne struktury wewnętrzne modelu przedstawiono na rys. \ref{anizotropia}.

\begin{figure}[ht]
\centering
\includegraphics[scale=0.5]{images/anisotropy.png}
\caption{Porównanie dwóch zastosowanych siatek w obiekcie przytwierdzonym górną podstawą i poddanemu sile grawitacji. Lewo: Anizotropia obserwowana w czworościennej siatce połączeń. Prawo: Brak anizotropii w sześciennej siatce połączeń. Źródło: \cite{ca}}
\label{anizotropia}
\end{figure}

Okazuje się, że kalibracja parametrów modelu nie jest trywialna. W \cite{usa} autorzy proponuję metodę kalibracji poszczególnych stałych sprężystości sprężyn. Ich metoda pozwala na symulację zarówno izotropowych jak i anizotropowych. Jednak jak sami autorzy wskazują jest złożona obliczeniowo,a w publikacji została zaprezentowany tylko przykład dla siatek dwuwymiarowych. Inne podejście w swojej publikacji przedstawili francuscy badacze wykorzystując do estymacji parametrów sprężystości algorytmy genetyczne.\cite{ei}

Alternatywne podejście, odchodzące nieco od klasycznego modelu punktów mas i sprężyn, zostało przedstawione w publikacji D. Bourguignon i M-P. Cani \cite{ca}. Metoda ta, pochodna od systemu punktów mas i sprężyn, pozwala na definiowanie własności mechanicznych symulowanych obiektów niezależnie od przyjętej geometrii czy topologii. Pozwala to na wczytanie do symulacji obiektów utworzonych w programach do modelowania 3D i co ważniejsze, odciążenia grafika z konieczności uwzględnienia podczas pracy fizycznych charakterystyk modelu.\cite{ca}

Metoda zakłada, wszystkie punkty masy w modelu są pogrupowane w tzw. jednostki objętości (volume element), którymi najczęściej są czworościany. Dla każdej jednostki wyznacza się tymczasowe punkty masy położone wzdłuż stałych, predefiniowanych osi. Umiejscowienie osi ma odzwierciedlać mechaniczną charakterystykę obiektu. Z reguły stosuje się standardowo 3 osie, jednak możliwa jest też ich większa ilość \cite{ca}. Sposób wyznaczania punktów przecięcia przedstawiony jest na rysunku \ref{anizotropia-czworoscian}.

\begin{figure}[ht]
\centering
\begin{tikzpicture}

\coordinate (A) at (0, 0, 0);
\coordinate (B) at (6, 0, 0);
\coordinate (C) at (3, 0, 4.22);
\coordinate (D) at (3, 4.89, 2.11);

\draw[-] (D) -- (B) -- (C) -- (D) -- (A) -- (C);
\draw[-, dashed] (A) -- (B);

\filldraw[fill=red, draw=black] (A) circle (3pt);
\filldraw[fill=red, draw=black] (B) circle (3pt);
\filldraw[fill=red, draw=black] (C) circle (3pt);
\filldraw[fill=red, draw=black] (D) circle (3pt);

\node[above left] at (D) {A};
\node[below right] at (B) {B};
\node[below left] at (C) {C};
\node[below left] at (A) {D};

%intersection points
\coordinate (I1) at (barycentric cs:D=0.5,B=0.7,C=0.5);
\coordinate (I2) at (barycentric cs:A=0.7,B=0.5,D=0.5);
\coordinate (Imid) at (barycentric cs:I1=0.5,I2=0.5);

\filldraw[fill=blue, draw=black] (I1) circle (3pt);
\filldraw[fill=blue, draw=black] (I2) circle (3pt);
\node[above right] at (I1) {$P_1$};
\node[above left] at (I2) {$P_2$};
\node[below=25pt, right] at (I1) {$\alpha$};
\node[below=15pt, left] at (I2) {$\beta$};
\node[right, above=20pt] at (I1) {$\gamma$};

\draw[-,snake=snake] (I1) -- (I2);
\draw[<->,thick, shorten >=4pt, shorten <=4pt] (I1) -- (I2);
\draw[->,thick] (Imid) -- ++(0,0.6,0);
\draw[->,thick] (Imid) -- ++(0,-0.6,0);

\draw[-,densely dotted] (I1) -- (C);
\draw[-,densely dotted] (I1) -- (D);
\draw[-,densely dotted] (I1) -- (B);


\end{tikzpicture}

\caption{Wyznaczanie punktu przecięcia z osiami w czworościanie.}
\label{anizotropia-czworoscian}
\end{figure}

Przedstawiony czworościan posiada zdefiniowane dwie osie. Wyznaczono też dwa punkty przecięcia $P_1$ oraz $P_2$ z osią poziomą figury. W celu zapamiętania pozycji punktów przecięcia wyznacza się współczynniki kombinacji liniowej z wierzchołkami tworzącymi ścianę. $P_1 = \alpha * A + \beta *B + \gamma *C$. Współczynniki te muszą być wyznaczane dla czworościanu znajdującego się w stanie spoczynku. Punkty przecięcia traktowane są odtąd jak nowe punkty masy. Działają na nie siły wewnętrzne i zewnętrzne układu. Dwa punkty (zaznaczonymi na rys. \ref{anizotropia-czworoscian} kolorem niebieskim) zostają w istocie połączone sprężyną.

Następnie dokonuje się omawianych w poprzednich podrozdziałach obliczeń sił działających na punkt przecięcia. Mając dane współczynniki kombinacji liniowej, wyznaczyć można siły działające na punkty masy $A,B,C$ pierwotnie zdefiniowane w modelu. 

\begin{figure}[ht]
\centering
\includegraphics[scale=0.5]{images/fixed_anisotropy.png}
\caption{Porównanie dwóch zastosowanych siatek w obiekcie przytwierdzonym górną podstawą i poddanemu sile grawitacji. W modelu wykorzystano metodę D. Bourguignon i M-P. Cani, Źródło: \cite{ca}}
\label{anizotropia-czworoscian-fix}
\end{figure}
