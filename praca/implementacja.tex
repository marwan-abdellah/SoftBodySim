\chapter{Implementacja}
W rozdziale tym przedstawiona zostanie implementacja opisanych w poprzednich
rozdziałach technik symulacji ciała miękkiego. Na implementację składają się dwa
główne komponenty - biblioteka programistyczna oraz aplikacja demonstracyjna.
Biblioteka programistyczna jest odpowiedzialna za symulację ciała miękkiego.
Stworzenie jej miało na celu ułatwienie ponownego
wykorzystania zaimplementowanych klas i umożliwienie jej samodzielnego
rozwijania w przyszłości.

Bibliotekę można skompilować w wersji tylko na CPU oraz ze wsparciem dla technologii
CUDA.  Proces ten jest automatyczny i wspierany przez użyty w projekcie program
CMake, służący do zarządzania kompilacją programu. W przypadku gdy w systemie
zostaną wykryte wymagane biblioteki oraz kompilator CUDA, nastąpi kompilacja
dodatkowych plików źródłowych.

Program CMake jest wieloplatformowy, co oznacza, że dostępne są jego wersje na
najpopularniejsze obecnie systemy operacyjne takie jak Windows, Linux czy Os X.
CMake potrafi generować pliki z regułami kompilacji dla konkretnego środowiska
deweloperskiego. Dla platformy Linux będą to skrypty \texttt{Makefile},
	natomiast dla Windows pliki projektowe \texttt{Microsoft Visual Studio}.
	Zastosowanie CMake pozwala w wygodny sposób udostępniać projekt na różne
	platformy.

% opengl
Zarówno biblioteka jak i aplikacja demonstracyjna wymagają do uruchomienia
biblioteki OpenGL w wersji minimalnie 3.3. Jest ona wymagana do uruchomienia
zaimplementowanych w bibliotece vertex, pixel i geometry shaderów. Do implementacji obliczeń
na procesorze graficznym użyto CUDA-SDK w wersji 5.5.

W implementacji wykorzystano też dwie pomocnicze biblioteki.
Pierwszą jest GLM (OpenGL Math), dostarczająca implementację wielu podstawowych
typów niezbędnych w grafice trójwymiarowej, takich jak wektor czy macierz oraz
operacji takich jak np. iloczyn skalarny czy wektorowy. Największą zaletą
biblioteki GLM jest fakt, że składa się ona tylko z plików nagłówkowych. Nie ma
zatem potrzeby wymagania jej przy procesie kompilacji czy budowania jej razem z
projektem. Wszystkie potrzebne funkcje są bezpośrednio załączane do kodu
biblioteki w momencie kompilacji.

Kolejną zaletą biblioteki GLM jest jej kompatybilność z technologią CUDA. W przypadku
gdy plik nagłówkowy biblioteki jest przetwarzany przez preprocesor kompilatora
nvcc, zdefiniowane tam makra są rozwijane na charakterystyczne dla nvcc
atrybuty kompilatora, takie jak \texttt{\_\_device\_\_} czy \texttt{\_\_host\_\_}.
W przypadku gdy użyty jest inny kompilator biblioteka GLM wykorzystuje inne
atrybuty. Umożliwia to zadeklarowanie jednej funkcji która może być użyta zarówno
w kodzie wykonywanym na GPU, jak również na CPU.

Drugą wykorzystaną biblioteką jest OpenGL Framework (GLFW) stanowiąca warstwę
abstrakcji nad natywny system zarządzania oknami oraz zdarzeniami z klawiatury i
myszki. Dostarcza też mechanizm pętli głównej niezbędny do stworzenia
interaktywnych aplikacji.

\section{Aplikacja demonstracyjna}
\subsection{Przewodnik}
Aplikacja demonstracyjna jest prostą aplikacją ze sterowaniem klawiaturowym. Po
uruchomieniu główny plan zajmuje pusta trójwymiarowa scena. Nawigacja w
aplikacji obydwa się za pośrednictwem prawego klawisza myszy i przemiszczenia
kursora. Użytkownik jest w stanie obracać się punktu
centralnego sceny oraz przybliżać się i oddalać od niego używając kółka myszki.

\begin{figure}[H]
\centering
\includegraphics[scale=0.5]{images/z1.jpg}
\caption{Charakterystyki procesorów firmy Intel na przestrzeni 40 lat. Źródło: http://www.gotw.ca/}
\end{figure}

Aby zainicjować symulację należy użyć klawisza "t". Zostanie w ten sposób
stworzony model żaby, której model na licencji Royality Free
dostępny jest na stronie http://www.turbosquid.com/FullPreview/Index.cfm/ID/589370.

\begin{figure}[H]
\centering
\includegraphics[scale=0.5]{images/z2.jpg}
\caption{Charakterystyki procesorów firmy Intel na przestrzeni 40 lat. Źródło: http://www.gotw.ca/}
\end{figure}

Program demonstracyjny posiada dwa tryby renderowania. W pierwszym wyświetlany
jest cały oteksturowany model wraz z oświetleniem, w drugim zaś renderowana jest
tylko siatka połączeń między wierzchołkami tworzącymi trójkąty. Aby przełączać
się między dwoma trybami należy użyć klawisza "m".

\begin{figure}[H]
\centering
\includegraphics[scale=0.5]{images/z3.jpg}
\caption{Charakterystyki procesorów firmy Intel na przestrzeni 40 lat. Źródło: http://www.gotw.ca/}
\end{figure}

W stworzonej aplikacji istnieje możliwość interakcji z symulowanym ciałem
miękkim. W tym celu należy najechać myszą na obszar zajmowany przez model a
następnie nacisnąć i trzymać lewy przycisk myszy.

\begin{figure}[H]
\centering
\includegraphics[scale=0.5]{images/z5.jpg}
\caption{Charakterystyki procesorów firmy Intel na przestrzeni 40 lat. Źródło: http://www.gotw.ca/}
\end{figure}

Możliwe jest również zmienianie współczynnika sprężystości symulowanego ciała. W
przypadku gdy jest on większy symulowane ciało zachowywać się będzie bardziej
jak bryła sztywna, natomiast dla małych wartości ciało wskazywać będzie zanik
wszelkich sił wewnętrznych. Zmiana sił sprężystości możliwa jest poprzez
klawisze '-' i '+'.

% GLWF
\subsection{Rysowanie okna}
W celu stworzenia okna aplikacji posłużono się open-sourcową biblioteką GLFW. Z
racji, że biblioteka te udostępnia interfejsy programisty napisane tylko w języku C
utworzona została klasa \texttt{GLFWApplication} opakowująca API biblioteki
GLFW (ang. wrapper). Oprócz zwykłego udostępniania metod, klasa ta spełnia
jeszcze dodatkowe funkcje. Udostępnia ona wirtualne metody \texttt{OnRender}
oraz \texttt{OnUpdate}, które będą wykorzystywane odpowiednio przy rysowaniu
ramki obrazu oraz przy symulacji fizyki. Dodatkowo klasa GLFWApplication rejestruje wszystkie błędy zgłaszane przez
GLFW i drukuje je do standardowego wyjścia błędów.

% statyczny krok symulacji
\subsection{Krok symulacji}
Kolejnymi funkcjonalnościami dostarczanymi przez klasę \texttt{GLFWApplication} jest
uruchomienie pętli głównej aplikacji oraz wywoływanie metod \texttt{OnUpdate} oraz \texttt{OnRender}.
Jest to niezbędne w celu zsynchronizowania odświeżania obrazu i
symulacji fizyki. Pętla główna zdefiniowana na listingu (\ref{main_loop})
pozwala na użycie stałego kroku czasu w symulacji fizyki przy jednoczesnym
zachowaniu możliwości odświeżania obrazu w różnej częstotliwości. Przyjęcie
stałego kroku czasu jest niezbędne w symulacji ponieważ
techniki w niej zaimplementowane są wrażliwe na jego zmianę, czego efektem jest
np. zmiana sztywności ciała obserwowana przy zmianie kroku czasu.

\begin{lstlisting}[caption=Pętla główna, label=main_loop]
void GLFWApplication::MainLoop(double delta)
{
	double accumulator = 0.0;
	lastFrameTime = glfwGetTime();

	while (!glfwWindowShouldClose(m_window)) {
		double currentTime = glfwGetTime();
		double frameTime = currentTime - lastFrameTime;

		if (frameTime > MAX_FRAME_TIME)
			frameTime = MAX_FRAME_TIME;

		accumulator += frameTime;
		lastFrameTime = currentTime;

		while (accumulator >= delta) {
			OnUpdate(delta);
			accumulator -= delta;
		}

		OnRender();
		glfwSwapBuffers(m_window);
		glfwPollEvents();
	}

	glfwDestroyWindow(m_window);
}
\end{lstlisting}

% łapanie obiektów
\subsection{Chwytanie obiektów}
W aplikacji demonstracyjnej możliwe jest ,,chwytanie'' części symulowanych
obiektów. Możliwe jest to dzięki funkcji z interfejsu programistycznemu
biblioteki (API). Jako parametry wejściowe funkcja ta przyjmuje półprostą (ang.
		Ray) oraz liczbę zmiennoprzecinkową oznaczającą maksymalną odległość
punktów od półprostej, które mają zostać ,,schwytane''. Operacja tworzenia
półprostej realizowana jest dwuetapowo. W pierwszym etapie ze współrzędnych
\texttt{x, y} myszki wyliczany jest odpowiadający im punkt w przestrzeni
trójwymiarowej, czyli tzw. współrzędnych świata (ang. world
		coordinates). Realizuje to funkcja \texttt{GetWorldCoordinates},
	zaimplementowana w klasie \texttt{Demo} dziedziczącej po klasie
	\texttt{GLFWApplication}. Jej implementacja
	przedstawiona jest na listingu poniżej:

\begin{lstlisting}[caption=Estymacja wektora w przstrzeni trójwymiarowej na
	podstawie pozycji myszki na ekranie, label=screen2world]
Ray Demo::GetRayFromCamera()
{
	glm::vec3 ret;
	glm::uvec2 mouse = GetMouseCoords();

	ret[0] = (2.0f * mouse[0]) / width - 1.0f;
	ret[1] = 1.0f - (2.0 * mouse[1]) / height;
	ret[2] = 0.0f;

	glm::vec4 pos = glm::vec4(ret, 1.0);
	pos = glm::inverse(renderer.GetProjectionMatrix()) * pos;

	pos[3] = 0.0f;

	pos = glm::inverse(mCamera.getCameraMatrix()) * pos;

	return Ray(mCamera.GetEyePosition(), pos.xyz());
}
\end{lstlisting}

Ideą funkcji przedstawionej na listingu \ref{screen2world} jest odwrócenie
procesu zachodzącego przy rysowaniu ramki obrazu przez bibliotekę OpenGL.
% dołożyć krótki schemat 
Na początku ze współrzędnych myszki tworzony jest trójwymiarowy wektor
wyrażony w tzw. znormalizowanych współrzędnych urządzenia (ang. Normalized
		Device Coordinates). Specyfikacja OpenGL zakłada, że współrzędne te zawierają się w
przedziale -1.0 do 1.0. Dodatkowo przy wyliczaniu drugiej współrzędnej
\texttt{ndc[1]} trzeba uwzględnić fakt, że biblioteka GLFW podaje współrzędne
myszki zaczynając od lewego górnego rogu okna, natomiast OpenGL zakłada, że
współrzędne (-1.0, -1.0) znajdują się w lewym dolnym rogu okna.

Trzecią koordynatą używaną w NDC jest wartość bufora głębokości. Może ona zostać
odczytana z bufora głębokości po wygenerowaniu ramki obrazu przy użyciu funkcji
OpenGL \texttt{glReadPixels}, jednak w przypadku gdy skonstruować chcemy tylko
półprostą każda wartość mieszcząca się przedziałach (-1, 1) będzie odpowiednia.
Kolejnym krokiem jest zamiana NDC do tzw. współrzędnych przycięcia (ang. clip
		coordinates). W tym celu musimy skonstruować czterowymiarowy wektor,
		 którego czwarta współrzędna jest współczynnikiem skalowania pozostałych
		 współrzędnych. Aby nie modyfikować wartości już wyliczonych
		 współrzędnych przyjmujemy że ostatnia współrzędna wektora jest równa 1.
		 
Kolejnym etapem jest odwrócenie wpływu macierzy projekcji. Dzieje się
poprzez przemnożenie wektora w współrzędnych przycięcia przez odwróconą
macierz projekcji. Ostatnim etapem jest odwrócenie wpływu macierzy
kamery, (jest ona odpowiednikiem macierzy OpenGL GLMODELVIEW).
Zanim jednak wektor zostanie przemnożony przez odwróconą macierz kamery,
	  ostatnia współrzędna $w$ wektora zostanie wyzerowana. Ma to za cel
	  wyeliminowanie translacji z przekształcenia, gdyż w ten sposób
	  otrzymalibyśmy finalną pozycję punktu we współrzędnych modelu.

\section{Biblioteka}
Biblioteka programistyczna jest najważniejszym i największym komponentem
stworzonym na potrzeby tej pracy. Implementuje ona metodę \texttt{Shape
	Matchingu} opisaną w rozdziale\ref{sec:shape}, wraz z jej modyfikacją wymienioną w
	podrozdziale \ref{ssec:region} oraz technikę zachowania
	objętości opisaną w rozdziale {\ref{sec:vol}.

Biblioteka napisana jest w języku C++ 

\begin{figure}[H]
\centering

\tikzset{
	basic/.style  = {draw, text width=4cm, drop shadow, font=\sffamily, rectangle},
	root/.style   = {basic, rounded corners=2pt, thin, align=center, fill=green!30},
	level 2/.style = {basic, rounded corners=6pt, thin,align=center, fill=green!60, text width=8em},
	level 3/.style = {basic, thin, align=left, fill=pink!60, text
					width=9.0em}
}
\begin{tikzpicture}[
  level 1/.style={sibling distance=55mm},
    edge from parent/.style={->,draw},
	  >=latex]

	  % root of the the initial tree, level 1
	  \node[root] {libsoftbody-engine}
	  % The first level, as children of the initial tree
	    child {node[level 2] (c1) {Fizyka}}
		  child {node[level 2] (c2) {Model}}
		    child {node[level 2] (c3) {Renderowanie}};

% The second level, relatively positioned nodes
\begin{scope}[every node/.style={level 3}]
\node [below of = c1, xshift=25pt] (c11) {SoftBodySolver};
\node [below of = c11] (c12) {CUDASoftBodySolver};
\node [below of = c12] (c13) {CPUSoftBodySolver};

\node [below of = c2, xshift=25pt] (c21) {SoftBody};
\node [below of = c21] (c22) {MeshData};
\node [below of = c22] (c23) {Material};
\node [below of = c23] (c24) {OBJParser};
\node [below of = c24] (c25) {OBJLexer};

\node [below of = c3, xshift=25pt] (c31) {Scene};
\node [below of = c31] (c32) {Shader};
\node [below of = c32] (c33) {VertexBuffer};
\node [below of = c33] (c34) {Camera};
\end{scope}

% lines from each level 1 node to every one of its "children"
\foreach \value in {1,...,3}
  \draw[->] (c1.192) |- (c1\value.west);

  \foreach \value in {1,...,5}
    \draw[->] (c2.192) |- (c2\value.west);

	\foreach \value in {1,...,4}
	  \draw[->] (c3.192) |- (c3\value.west);

\end{tikzpicture}

\caption{Stos CUDA. Źródło: Opracowanie własne.}
\label{cuda-model}
\end{figure}

\subsection{Model w formacie Wavefront}
% format obj - zalety w sb
% opis klas lekser i parser, mesh data
% opis formatu
% duplikowanie verteksów

\subsection{Renderowanie}
% opis klasy renderowania
% cienie i generacja normalnych

% stabilność numeryczna
\subsection{Solwer CPU}

\subsection{Solwer GPU}
 % opis klasy 
 % preprocessing
 % updatowanie verteksów
\subsection{Stabilność numeryczna}

